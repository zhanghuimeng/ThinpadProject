% !Mode:: "TeX:UTF-8"
\documentclass{article}
\usepackage[hyperref, UTF8]{ctex}
\usepackage[dvipsnames]{xcolor}
\usepackage[top=1in, bottom=1in, left=0.8in, right=0.8in]{geometry}
\usepackage{amsmath}
\usepackage{amsfonts}
\usepackage{listings}
\usepackage[section]{placeins} % 避免浮动体越过subsection
\usepackage{pgfplotstable}
\usepackage{pgfplots}
\usepackage{fontspec}
\usepackage{underscore} % 使用下划线
\usepackage[english]{babel} % bug fix
\usepackage{booktabs} % 表格上的不同横线
\usepackage{supertabular} % 分页表格
\usepackage{comment} % 用于插入大段注释(喵喵喵?还有这种操作?)
\setmonofont[Mapping={}]{Consolas}	%英文引号之类的正常显示,相当于设置英文字体
\setsansfont{Consolas} %设置英文字体 Monaco, Consolas,  Fantasque Sans Mono
\setmainfont{Consolas} %设置英文字体

\definecolor{mygreen}{rgb}{0,0.6,0}
\definecolor{mygray}{rgb}{0.5,0.5,0.5}
\definecolor{mymauve}{rgb}{0.58,0,0.82}
\lstset{ %
    backgroundcolor=\color{white},   % choose the background color
    basicstyle=\footnotesize\ttfamily,        % size of fonts used for the code
    columns=fullflexible,
    breaklines=true,                 % automatic line breaking only at whitespace
    captionpos=b,                    % sets the caption-position to bottom
    tabsize=4,
    backgroundcolor=\color[RGB]{245,245,244},            % 设定背景颜色
    commentstyle=\color{mygreen},    % comment style
    escapeinside={\%*}{*)},          % if you want to add LaTeX within your code
    keywordstyle=\color{blue},       % keyword style
    stringstyle=\color{mymauve}\ttfamily,     % string literal style
    showstringspaces=false,                % 不显示字符串中的空格
    frame=none,
    rulesepcolor=\color{red!20!green!20!blue!20},
    % identifierstyle=\color{red},
    language=c++,
}

% 设置hyperlink的颜色
\newcommand\myshade{85}
\colorlet{mylinkcolor}{violet}
\colorlet{mycitecolor}{YellowOrange}
\colorlet{myurlcolor}{Aquamarine}

\hypersetup{
  linkcolor  = mylinkcolor!\myshade!black,
  citecolor  = mycitecolor!\myshade!black,
  urlcolor   = myurlcolor!\myshade!black,
  colorlinks = true,
}

% 可以使用\newtag{word}{label name}来定义一个table中的label
\makeatletter
\newcommand\newtag[2]{#1\def\@currentlabel{#1}\label{#2}}
\makeatother

% 如何好地使用nameref
\makeatletter
\newcommand{\labelname}[1]{% \labelname{<stuff>}
  \def\@currentlabelname{#1}}%
\makeatother

\title{CPU各模块及接口说明}
\author{喵喵喵喵喵?}

\begin{document}

\begin{comment}
普通的tabular环境:
\begin{table}
    \centering
    \small
    \begin{tabular}{lllllp{2cm}}
    \toprule
    方向 & 名称 & 类型 & 宽度 & 连接到 & 详细描述 \\ \midrule
    in & rst\label{REGISTERS:rst} & STD_LOGIC & 1 & \nameref{sec:MIPS_CPU} & 复位信号 \\
    in & clk\label{REGISTERS:clk} & STD_LOGIC & 1 & \nameref{sec:MIPS_CPU} & 时钟信号 \\

    \bottomrule
    \end{tabular}
    \caption {PC的接口}
\end{table}

可以换页的tabular环境:
\begin{center}
    \tablecaption{PC的接口}

    \tablefirsthead{
        \toprule
        \multicolumn{1}{l}{方向} &
        \multicolumn{1}{l}{名称} &
        \multicolumn{1}{l}{类型} &
        \multicolumn{1}{l}{宽度} &
        \multicolumn{1}{l}{连接到} &
        \multicolumn{1}{l}{详细描述} \\
        \midrule }

    \tablehead{
        \toprule
        \multicolumn{1}{l}{方向} &
        \multicolumn{1}{l}{名称} &
        \multicolumn{1}{l}{类型} &
        \multicolumn{1}{l}{宽度} &
        \multicolumn{1}{l}{连接到} &
        \multicolumn{1}{l}{详细描述} \\
        \midrule }

    \tabletail{
        \bottomrule
        \multicolumn{6}{c}{接下页} \\}

    \tablelasttail{\bottomrule}

    \small
    \begin{supertabular}{lllllp{2cm}}

    in & rst\label{REGISTERS:rst} & STD_LOGIC & 1 & \nameref{sec:MIPS_CPU} & 复位信号 \\
    in & clk\label{REGISTERS:clk} & STD_LOGIC & 1 & \nameref{sec:MIPS_CPU} & 时钟信号 \\

    \end{supertabular}
\end{center}

好的使用表格内label的方法:
\labelname{name}\newtag{word}{ref:label}
\end{comment}

\maketitle

\tableofcontents
\newpage

\section{PC\label{sec:PC}}

\subsection{简介}

取指模块。然后好像没什么好说的了。

\FloatBarrier
\subsection{接口定义}

\begin{center}
    \tablecaption{PC模块的接口}

    \tablefirsthead{
        \toprule
        \multicolumn{1}{l}{方向} &
        \multicolumn{1}{l}{名称} &
        \multicolumn{1}{l}{类型} &
        \multicolumn{1}{l}{宽度} &
        \multicolumn{1}{l}{连接到} &
        \multicolumn{1}{l}{详细描述} \\
        \midrule }

    \tablehead{
        \toprule
        \multicolumn{1}{l}{方向} &
        \multicolumn{1}{l}{名称} &
        \multicolumn{1}{l}{类型} &
        \multicolumn{1}{l}{宽度} &
        \multicolumn{1}{l}{连接到} &
        \multicolumn{1}{l}{详细描述} \\
        \midrule }

    \tabletail{
        \bottomrule
        \multicolumn{6}{c}{接下页} \\}

    \tablelasttail{\bottomrule}

    \small
    \begin{supertabular}{llllp{3.5cm}p{2cm}}
    in & \labelname{rst}\newtag{rst}{PC:rst} & STD_LOGIC & 1 & \nameref{sec:MIPS_CPU}.\nameref{MIPS_CPU:rst} & 复位信号 \\
    in & \labelname{clk}\newtag{clk}{PC:clk} & STD_LOGIC & 1 & \nameref{sec:MIPS_CPU}.\nameref{MIPS_CPU:clk} & 时钟信号 \\
    in & \labelname{pause_i}\newtag{pause_i}{PC:pause_i} & STD_LOGIC_VECTOR & \nameref{const:CTRL_PAUSE_LEN} & \nameref{sec:PAUSE_CTRL}.\nameref{PAUSE_CTRL:pause_o} & 此模块是否暂停 \\
    in & \labelname{branch_i}\newtag{branch_i}{PC:branch_i} & STD_LOGIC & 1 & \nameref{sec:ID}.\nameref{ID:branch_o} & 是否跳转 \\
    in & \labelname{branch_target_addr_i}\newtag{branch_target_addr_i}{PC:branch_target_addr_i} & STD_LOGIC_VECTOR  & \nameref{const:INST_ADDR_LEN}  & \nameref{sec:ID}.\nameref{ID:branch_target_addr_o} & 如果跳转,跳到什么位置    \\
    out & \labelname{en_o}\newtag{en_o}{PC:en_o} & STD_LOGIC & 1 & \nameref{sec:MIPS_CPU}.\nameref{MIPS_CPU:rom_en_o} & 是否读指令 \\
    out & \labelname{pc_o}\newtag{pc_o}{PC:pc_o} & STD_LOGIC_VECTOR & \nameref{const:INST_ADDR_LEN}  & \nameref{sec:MIPS_CPU}.\nameref{MIPS_CPU:rom_addr_o}, \nameref{sec:IF/ID}.\nameref{IF/ID:pc_i} & 下一条指令的位置 \\

    \end{supertabular}
\end{center}
\FloatBarrier

\section{IF/ID\label{sec:IF/ID}}

\subsection{简介}

\FloatBarrier
\subsection{接口定义}

\begin{center}
    \tablecaption{IF/ID模块的接口}

    \tablefirsthead{
        \toprule
        \multicolumn{1}{l}{方向} &
        \multicolumn{1}{l}{名称} &
        \multicolumn{1}{l}{类型} &
        \multicolumn{1}{l}{宽度} &
        \multicolumn{1}{l}{连接到} &
        \multicolumn{1}{l}{详细描述} \\
        \midrule }

    \tablehead{
        \toprule
        \multicolumn{1}{l}{方向} &
        \multicolumn{1}{l}{名称} &
        \multicolumn{1}{l}{类型} &
        \multicolumn{1}{l}{宽度} &
        \multicolumn{1}{l}{连接到} &
        \multicolumn{1}{l}{详细描述} \\
        \midrule }

    \tabletail{
        \bottomrule
        \multicolumn{6}{c}{接下页} \\}

    \tablelasttail{\bottomrule}

    \small
    \begin{supertabular}{llllp{3.5cm}p{2cm}}

    in & \labelname{rst}\newtag{rst}{IF/ID:rst} & STD_LOGIC & 1 & \nameref{sec:MIPS_CPU}.\nameref{MIPS_CPU:rst} & 复位信号 \\
    in & \labelname{clk}\newtag{clk}{IF/ID:clk} & STD_LOGIC & 1 & \nameref{sec:MIPS_CPU}.\nameref{MIPS_CPU:clk} & 时钟信号 \\
    in & \labelname{pc_i}\newtag{pc_i}{IF/ID:pc_i} & STD_LOGIC_VECTOR & \nameref{const:INST_ADDR_LEN} & \nameref{sec:PC}.\nameref{PC:pc_o} & 指令地址 \\
    in & \labelname{inst_i}\newtag{inst_i}{IF/ID:inst_i} & STD_LOGIC_VECTOR & \nameref{const:INST_LEN} & \nameref{sec:MIPS_CPU}.\nameref{MIPS_CPU:inst_i} & 指令 \\
    in & \labelname{pause_i}\newtag{pause_i}{IF/ID:pause_i} & STD_LOGIC_VECTOR & \nameref{const:CTRL_PAUSE_LEN} & \nameref{sec:PAUSE_CTRL}.\nameref{PAUSE_CTRL:pause_o} & 是否暂停 \\
    out & \labelname{pc_o}\newtag{pc_o}{IF/ID:pc_o} & STD_LOGIC_VECTOR & \nameref{const:INST_ADDR_LEN} & \nameref{sec:ID}.\nameref{ID:pc_i} & 指令地址 \\
    out & \labelname{inst_o}\newtag{inst_o}{IF/ID:inst_o} & STD_LOGIC_VECTOR & \nameref{const:INST_LEN} & \nameref{sec:ID}.\nameref{ID:inst_i}  & 指令 \\

    \end{supertabular}
\end{center}
\FloatBarrier

\section{ID}
\label{sec:ID}

\subsection{简介}

译码模块。主要工作流程是,先根据op确定指令类型,再根据special_funct确定具体是什么指令,然后根据指令类型进行译码。
\begin{description}
  \item [SPECIAL类指令]

    \begin{description}
      \item[SHIFT类指令] 读rt,扩展shamt,写rd
      \item[待写]
    \end{description}

  \item 待写
\end{description}

在译码结束之后,最后还有一段,根据EX和MEM阶段写寄存器的情况解决数据冲突,然后确定指令对应的两个操作数operand_1_o和operand_2_o。

\FloatBarrier
\subsection{接口定义}

\begin{center}
    \tablecaption{ID模块的接口}

    \tablefirsthead{
        \toprule
        \multicolumn{1}{l}{方向} &
        \multicolumn{1}{l}{名称} &
        \multicolumn{1}{l}{类型} &
        \multicolumn{1}{l}{宽度} &
        \multicolumn{1}{l}{连接到} &
        \multicolumn{1}{l}{详细描述} \\
        \midrule }

    \tablehead{
        \toprule
        \multicolumn{1}{l}{方向} &
        \multicolumn{1}{l}{名称} &
        \multicolumn{1}{l}{类型} &
        \multicolumn{1}{l}{宽度} &
        \multicolumn{1}{l}{连接到} &
        \multicolumn{1}{l}{详细描述} \\
        \midrule }

    \tabletail{
        \bottomrule
        \multicolumn{6}{c}{接下页} \\}

    \tablelasttail{\bottomrule}

    \small
    \begin{supertabular}{lp{3cm}llp{4cm}p{2cm}}
    in & \labelname{rst}\newtag{rst}{ID:rst} & STD_LOGIC & 1 & \nameref{sec:MIPS_CPU}.\nameref{MIPS_CPU:rst} & 复位信号 \\
    in & \labelname{pc_i}\newtag{pc_i}{ID:pc_i} & STD_LOGIC_VECTOR & \nameref{const:INST_ADDR_LEN} & \nameref{sec:IF/ID}.\nameref{IF/ID:pc_o} & 指令地址 \\
    in & \labelname{inst_i}\newtag{inst_i}{ID:inst_i} & STD_LOGIC_VECTOR & \nameref{const:INST_LEN} & \nameref{sec:IF/ID}.\nameref{IF/ID:inst_o} & 指令 \\
    in & \labelname{reg_rd_data_1_i}\newtag{reg_rd_data_1_i}{ID:reg_rd_data_1_i} & STD_LOGIC_VECTOR & \nameref{const:REG_DATA_LEN} & \nameref{sec:REGISTERS}.\nameref{REGISTERS:reg_rd_data_1_o} & 寄存器1读出数据 \\
    in & \labelname{reg_rd_data_2_i}\newtag{reg_rd_data_2_i}{ID:reg_rd_data_2_i} & STD_LOGIC_VECTOR & \nameref{const:REG_DATA_LEN} & \nameref{sec:REGISTERS}.\nameref{REGISTERS:reg_rd_data_2_o} & 寄存器2读出数据 \\
    in & \labelname{ex_reg_wt_en_i}\newtag{ex_reg_wt_en_i}{ID:ex_reg_wt_en_i} & STD_LOGIC & 1 & \nameref{sec:EX}.\nameref{EX:reg_wt_en_o} & EX 模块是否写寄存器 \\
    in & \labelname{ex_reg_wt_addr_i}\newtag{ex_reg_wt_addr_i}{ID:ex_reg_wt_addr_i} & STD_LOGIC_VECTOR & \nameref{const:REG_ADDR_LEN} & \nameref{sec:EX}.\nameref{EX:reg_wt_addr_o} & EX 模块写寄存器地址 \\
    in & \labelname{ex_reg_wt_data_i}\newtag{ex_reg_wt_data_i}{ID:ex_reg_wt_data_i} & STD_LOGIC_VECTOR & \nameref{const:REG_DATA_LEN} & \nameref{sec:EX}.\nameref{EX:reg_wt_data_o} & EX 模块写寄存器数据 \\
    in & \labelname{mem_reg_wt_en_i}\newtag{mem_reg_wt_en_i}{ID:mem_reg_wt_en_i} & STD_LOGIC & 1 & \nameref{sec:MEM}.\nameref{MEM:reg_wt_en_o} & MEM 模块是否写寄存器 \\
    in & \labelname{mem_reg_wt_addr_i}\newtag{mem_reg_wt_addr_i}{ID: mem_reg_wt_addr_i} & STD_LOGIC_VECTOR & \nameref{const:REG_ADDR_LEN} & \nameref{sec:MEM}.\nameref{MEM:reg_wt_addr_o} & MEM 模块写寄存器地址 \\
    in & \labelname{mem_reg_wt_data_i}\newtag{mem_reg_wt_data_i}{ID:mem_reg_wt_data_i} & STD_LOGIC_VECTOR & \nameref{const:REG_DATA_LEN} & \nameref{sec:MEM}.\nameref{MEM:reg_wt_data_o} & MEM 模块写寄存器数据 \\
    in & \labelname{is_in_delayslot_i}\newtag{is_in_delayslot_i}{ID:is_in_delayslot_i} & STD_LOGIC & 1 & \nameref{sec:ID/EX}.\nameref{ID/EX:is_in_delayslot_o} & 当前指令是否在延迟槽内 \\
    out & \labelname{op_o}\newtag{op_o}{ID:op_o} & STD_LOGIC_VECTOR & \nameref{const:OP_LEN} & \nameref{sec:ID/EX}.\nameref{ID/EX:op_i} & 指令操作类型 \\
    out & \labelname{funct_o}\newtag{funct_o}{ID:funct_o} & STD_LOGIC_VECTOR & \nameref{const:FUNCT_LEN} & \nameref{sec:ID/EX}.\nameref{ID/EX:funct_i} & 指令子操作类型 \\
    out & \labelname{reg_rd_en_1_o}\newtag{reg_rd_en_1_o}{ID:reg_rd_en_1_o} & STD_LOGIC & 1 & \nameref{sec:REGISTERS}.\nameref{REGISTERS:reg_rd_en_1_i} & 寄存器1 读使能 \\
    out & \labelname{reg_rd_en_2_o}\newtag{reg_rd_en_2_o}{ID:reg_rd_en_2_o} & STD_LOGIC & 1 & \nameref{sec:REGISTERS}.\nameref{REGISTERS:reg_rd_en_2_i} & 寄存器2读使能 \\
    out & \labelname{reg_rd_addr_1_o}\newtag{reg_rd_addr_1_o}{ID:reg_rd_addr_1_o} & STD_LOGIC_VECTOR & \nameref{const:REG_ADDR_LEN} & \nameref{sec:REGISTERS}.\nameref{REGISTERS:reg_rd_addr_1_i} & 寄存器1读地址 \\
    out & \labelname{reg_rd_addr_2_o}\newtag{reg_rd_addr_2_o}{ID:reg_rd_addr_2_o} & STD_LOGIC_VECTOR & \nameref{const:REG_ADDR_LEN} & \nameref{sec:REGISTERS}.\nameref{REGISTERS:reg_rd_addr_2_i} & 寄存器2读地址 \\
    out & \labelname{operand_1_o}\newtag{operand_1_o}{ID:operand_1_o} & STD_LOGIC_VECTOR & \nameref{const:DATA_LEN} & \nameref{sec:ID/EX}.\nameref{ID/EX:operand_1_i} & 指令操作数1 \\
    out & \labelname{operand_2_o}\newtag{operand_2_o}{ID:operand_2_o} & STD_LOGIC_VECTOR & \nameref{const:DATA_LEN} & \nameref{sec:ID/EX}.\nameref{ID/EX:operand_2_i} & 指令操作数2 \\
    out & \labelname{extended_offset_o}\newtag{extended_offset_o}{ID:extended_offset_o} & STD_LOGIC_VECTOR & \nameref{const:DATA_LEN} & \nameref{sec:ID/EX}.\nameref{ID/EX:extended_offset_i} & 扩展后立即数 \\
    out & \labelname{reg_wt_en_o}\newtag{reg_wt_en_o}{ID:reg_wt_en_o} & STD_LOGIC & 1 & \nameref{sec:ID/EX}.\nameref{ID/EX:reg_wt_en_i} & 寄存器写使能 \\
    out & \labelname{reg_wt_addr_o}\newtag{reg_wt_addr_o}{ID:reg_wt_addr_o} & STD_LOGIC_VECTOR & \nameref{const:REG_ADDR_LEN} & \nameref{sec:ID/EX}.\nameref{ID/EX:reg_wt_addr_i} & 寄存器写地址 \\
    out & \labelname{pause_o}\newtag{pause_o}{ID:pause_o} & STD_LOGIC & 1 & \nameref{sec:PAUSE_CTRL}.\nameref{PAUSE_CTRL:id_pause_i} & 是否需要暂停 \\
    out & \labelname{branch_o}\newtag{branch_o}{ID:branch_o} & STD_LOGIC & 1 & \nameref{sec:PC}.\nameref{PC:branch_i} & 当前是否为分支跳转指令 \\
    out & \labelname{branch_target_addr_o}\newtag{branch_target_addr_o}{ID:branch_target_addr_o} & STD_LOGIC_VECTOR & \nameref{const:INST_ADDR_LEN} & \nameref{sec:PC}.\nameref{PC:branch_target_addr_i} & 跳转地址 \\
    out & \labelname{is_in_delayslot_o}\newtag{is_in_delayslot_o}{ID:is_in_delayslot_o} & STD_LOGIC & 1 & \nameref{sec:ID/EX}.\nameref{ID/EX:is_in_delayslot_i} & 当前指令是否在延迟槽内 \\
    out & \labelname{next_inst_in_delayslot_o}\newtag{next_inst_in_delayslot_o}{ID:next_inst_in_delayslot_o} & STD_LOGIC & 1 & \nameref{sec:ID/EX}.\nameref{ID/EX:next_inst_in_delayslot_i} & 下一条指令是否在延迟槽内 \\
    out & \labelname{link_addr_o}\newtag{link_addr_o}{ID:link_addr_o} & STD_LOGIC_VECTOR & \nameref{const:INST_ADDR_LEN} & \nameref{sec:ID/EX}.\nameref{ID/EX:link_addr_i} & 跳转指令的返回地址 \\
    \end{supertabular}
\end{center}
\FloatBarrier

\section{ID/EX}
\label{sec:ID/EX}

\subsection{简介}

\FloatBarrier
\subsection{接口定义}

\begin{center}
    \tablecaption{ID/EX模块的接口}

    \tablefirsthead{
        \toprule
        \multicolumn{1}{l}{方向} &
        \multicolumn{1}{l}{名称} &
        \multicolumn{1}{l}{类型} &
        \multicolumn{1}{l}{宽度} &
        \multicolumn{1}{l}{连接到} &
        \multicolumn{1}{l}{详细描述} \\
        \midrule }

    \tablehead{
        \toprule
        \multicolumn{1}{l}{方向} &
        \multicolumn{1}{l}{名称} &
        \multicolumn{1}{l}{类型} &
        \multicolumn{1}{l}{宽度} &
        \multicolumn{1}{l}{连接到} &
        \multicolumn{1}{l}{详细描述} \\
        \midrule }

    \tabletail{
        \bottomrule
        \multicolumn{6}{c}{接下页} \\}

    \tablelasttail{\bottomrule}

    \small
    \begin{supertabular}{lp{2.5cm}llp{4cm}p{2cm}}

    in & \labelname{rst}\newtag{rst}{ID/EX:rst} & STD_LOGIC & 1 & \nameref{sec:MIPS_CPU}.\nameref{MIPS_CPU:rst} & 复位信号 \\
    in & \labelname{clk}\newtag{clk}{ID/EX:clk} & STD_LOGIC & 1 & \nameref{sec:MIPS_CPU}.\nameref{MIPS_CPU:clk} & 时钟信号 \\
    in & \labelname{op_i}\newtag{op_i}{ID/EX:op_i} & STD_LOGIC_VECTOR & \nameref{const:OP_LEN} & \nameref{sec:ID}.\nameref{ID:op_o} & 指令操作类型 \\
    in & \labelname{funct_i}\newtag{funct_i}{ID/EX:funct_i} & STD_LOGIC_VECTOR & \nameref{const:FUNCT_LEN} & \nameref{sec:ID}.\nameref{ID:funct_o} & 指令子操作类型 \\
    in & \labelname{operand_1_i}\newtag{operand_1_i}{ID/EX:operand_1_i} & STD_LOGIC_VECTOR & \nameref{const:REG_DATA_LEN} & \nameref{sec:ID}.\nameref{ID:operand_1_o} & 指令操作数1 \\
    in & \labelname{operand_2_i}\newtag{operand_2_i}{ID/EX:operand_2_i} & STD_LOGIC_VECTOR & \nameref{const:REG_DATA_LEN} & \nameref{sec:ID}.\nameref{ID:operand_2_o} & 指令操作数2 \\
    in & \labelname{extended_offset_i}\newtag{extended_offset_i}{ID/EX:extended_offset_i} & STD_LOGIC_VECTOR & \nameref{const:DATA_LEN} & \nameref{sec:ID}.\nameref{ID:extended_offset_o} & 扩展后立即数 \\
    in & \labelname{reg_wt_en_i}\newtag{reg_wt_en_i}{ID/EX:reg_wt_en_i} & STD_LOGIC & 1 & \nameref{sec:ID}.\nameref{ID:reg_wt_en_o} & 寄存器写使能 \\
    in & \labelname{reg_wt_addr_i}\newtag{reg_wt_addr_i}{ID/EX:reg_wt_addr_i} & STD_LOGIC_VECTOR & \nameref{const:REG_ADDR_LEN} & \nameref{sec:ID}.\nameref{ID:reg_wt_addr_o} & 寄存器写地址 \\
    in & \labelname{pause_i}\newtag{pause_i}{ID/EX:pause_i} & STD_LOGIC_VECTOR & \nameref{const:CTRL_PAUSE_LEN} & \nameref{sec:PAUSE_CTRL}.\nameref{PAUSE_CTRL:pause_o} & 是否暂停 \\
    in & \labelname{is_in_delayslot_i}\newtag{is_in_delayslot_i}{ID/EX:is_in_delayslot_i} & STD_LOGIC & 1 & \nameref{sec:ID}.\nameref{ID:is_in_delayslot_o} & 当前指令是否在延迟槽中 \\
    in & \labelname{next_inst_in_delayslot_i}\newtag{next_inst_in_delayslot_i}{ID/EX:next_inst_in_delayslot_i} & STD_LOGIC & 1 & \nameref{sec:ID}.\nameref{ID:next_inst_in_delayslot_o} & 下一条指令是否在延迟槽中 \\
    in & \labelname{link_addr_i}\newtag{link_addr_i}{ID/EX:link_addr_i} & STD_LOGIC_VECTOR & \nameref{const:INST_ADDR_LEN} & \nameref{sec:ID}.\nameref{ID:link_addr_o} & 跳转指令的返回地址 \\
    out & \labelname{op_o}\newtag{op_o}{ID/EX:op_o} & STD_LOGIC_VECTOR & \nameref{const:OP_LEN} & \nameref{sec:EX}.\nameref{EX:op_i} & 指令操作类型 \\
    out & \labelname{funct_o}\newtag{funct_o}{ID/EX:funct_o} & STD_LOGIC_VECTOR & \nameref{const:FUNCT_LEN} & \nameref{sec:EX}.\nameref{EX:funct_i} & 指令子操作类型 \\
    out & \labelname{operand_1_o}\newtag{operand_1_o}{ID/EX:operand_1_o} & STD_LOGIC_VECTOR & \nameref{const:REG_DATA_LEN} & \nameref{sec:EX}.\nameref{EX:operand_1_i} & 指令操作数1 \\
    out & \labelname{operand_2_o}\newtag{operand_2_o}{ID/EX:operand_2_o} & STD_LOGIC_VECTOR & \nameref{const:REG_DATA_LEN} & \nameref{sec:EX}.\nameref{EX:operand_2_i} & 指令操作数2 \\
    out & \labelname{extended_offset_o}\newtag{extended_offset_o}{ID/EX:extended_offset_o} & STD_LOGIC_VECTOR & \nameref{const:DATA_LEN} & \nameref{sec:EX}.\nameref{EX:extended_offset_i} & 扩展后立即数 \\
    out & \labelname{reg_wt_en_o}\newtag{reg_wt_en_o}{ID/EX:reg_wt_en_o} & STD_LOGIC & 1 & \nameref{sec:EX}.\nameref{EX:reg_wt_en_i} & 寄存器写使能 \\
    out & \labelname{reg_wt_addr_o}\newtag{reg_wt_addr_o}{ID/EX:reg_wt_addr_o} & STD_LOGIC_VECTOR & \nameref{const:REG_ADDR_LEN} & \nameref{sec:EX}.\nameref{EX:reg_wt_addr_i} & 寄存器写地址 \\
    out & \labelname{is_in_delayslot_o}\newtag{is_in_delayslot_o}{ID/EX:is_in_delayslot_o} & STD_LOGIC & 1 & \nameref{sec:EX}.\nameref{EX:is_in_delayslot_i} & 当前指令是否在延迟槽中 \\
    out & \labelname{next_inst_in_delayslot_o}\newtag{next_inst_in_delayslot_o}{ID/EX:next_inst_in_delayslot_o} & STD_LOGIC & 1 & \nameref{sec:ID}.\nameref{ID:is_in_delayslot_i} & 下一条指令是否在延迟槽中 \\
    out & \labelname{link_addr_o}\newtag{link_addr_o}{ID/EX:link_addr_o} & STD_LOGIC_VECTOR & \nameref{const:INST_ADDR_LEN} & \nameref{sec:EX}.\nameref{EX:link_addr_i} & 跳转指令的返回地址 \\
    \end{supertabular}
\end{center}
\FloatBarrier

\section{EX\label{sec:EX}}

\subsection{简介}

\FloatBarrier
\subsection{接口定义}

\begin{center}
    \tablecaption{EX模块的接口}

    \tablefirsthead{
        \toprule
        \multicolumn{1}{l}{方向} &
        \multicolumn{1}{l}{名称} &
        \multicolumn{1}{l}{类型} &
        \multicolumn{1}{l}{宽度} &
        \multicolumn{1}{l}{连接到} &
        \multicolumn{1}{l}{详细描述} \\
        \midrule }

    \tablehead{
        \toprule
        \multicolumn{1}{l}{方向} &
        \multicolumn{1}{l}{名称} &
        \multicolumn{1}{l}{类型} &
        \multicolumn{1}{l}{宽度} &
        \multicolumn{1}{l}{连接到} &
        \multicolumn{1}{l}{详细描述} \\
        \midrule }

    \tabletail{
        \bottomrule
        \multicolumn{6}{c}{接下页} \\}

    \tablelasttail{\bottomrule}

    \small
    \begin{supertabular}{llllp{4cm}p{2cm}}

    in & \labelname{rst}\newtag{rst}{EX:rst} & STD_LOGIC & 1 & \nameref{sec:MIPS_CPU}.\nameref{MIPS_CPU:rst} & 复位信号 \\
    in & \labelname{op_i}\newtag{op_i}{EX:op_i} & STD_LOGIC_VECTOR & \nameref{const:OP_LEN} & \nameref{sec:ID/EX}.\nameref{ID/EX:op_o} & 指令操作类型 \\
    in & \labelname{funct_i}\newtag{funct_i}{EX:funct_i} & STD_LOGIC_VECTOR & \nameref{const:FUNCT_LEN} & \nameref{sec:ID/EX}.\nameref{ID/EX:funct_o} & 指令子操作类型 \\
    in & \labelname{operand_1_i}\newtag{operand_1_i}{EX:operand_1_i} & STD_LOGIC_VECTOR & \nameref{const:REG_DATA_LEN} & \nameref{sec:ID/EX}.\nameref{ID/EX:operand_1_o} & 指令操作数1 \\
    in & \labelname{operand_2_i}\newtag{operand_2_i}{EX:operand_2_i} & STD_LOGIC_VECTOR & \nameref{const:REG_DATA_LEN} & \nameref{sec:ID/EX}.\nameref{ID/EX:operand_2_o} & 指令操作数2 \\
    in & \labelname{extended_offset_i}\newtag{extended_offset_i}{EX:extended_offset_i} & STD_LOGIC_VECTOR & \nameref{const:DATA_LEN} & \nameref{sec:ID/EX}.\nameref{ID/EX:extended_offset_o} & 扩展后立即数 \\
    in & \labelname{reg_wt_en_i}\newtag{reg_wt_en_i}{EX:reg_wt_en_i} & STD_LOGIC & 1 & \nameref{sec:ID/EX}.\nameref{ID/EX:reg_wt_en_o} & 寄存器写使能 \\
    in & \labelname{reg_wt_addr_i}\newtag{reg_wt_addr_i}{EX:reg_wt_addr_i} & STD_LOGIC_VECTOR & \nameref{const:REG_ADDR_LEN} & \nameref{sec:ID/EX}.\nameref{ID/EX:reg_wt_addr_o} & 寄存器写地址 \\
    in & \labelname{hi_i}\newtag{hi_i}{EX:hi_i} & STD_LOGIC_VECTOR & \nameref{const:REG_DATA_LEN} & \nameref{sec:HI_LO}.\nameref{HI_LO:hi_o} & HI寄存器 \\
    in & \labelname{lo_i}\newtag{lo_i}{EX:lo_i} & STD_LOGIC_VECTOR & \nameref{const:REG_DATA_LEN} & \nameref{sec:HI_LO}.\nameref{HI_LO:lo_o} & LO寄存器 \\
    in & \labelname{mem_hilo_en_i}\newtag{mem_hilo_en_i}{EX:mem_hilo_en_i} & STD_LOGIC & 1 & \nameref{sec:MEM}.\nameref{MEM:hilo_en_o} & MEM阶段的指令是否写HILO \\
    in & \labelname{mem_hi_i}\newtag{mem_hi_i}{EX:mem_hi_i} & STD_LOGIC_VECTOR & \nameref{const:REG_DATA_LEN} & \nameref{sec:MEM}.\nameref{MEM:hi_o} & MEM阶段的指令写HI的数据 \\
    in & \labelname{mem_lo_i}\newtag{mem_lo_i}{EX:mem_lo_i} & STD_LOGIC_VECTOR & \nameref{const:REG_DATA_LEN} & \nameref{sec:MEM}.\nameref{MEM:lo_o} & MEM阶段的指令写LO 的数据 \\
    in & \labelname{wb_hilo_en_i}\newtag{wb_hilo_en_i}{EX:wb_hilo_en_i} & STD_LOGIC & 1 & \nameref{sec:MEM/WB}.\nameref{MEM/WB:hilo_en_o} & WB 阶段的指令是否写HILO \\
    in & \labelname{wb_hi_i}\newtag{wb_hi_i}{EX:wb_hi_i} & STD_LOGIC_VECTOR & \nameref{const:REG_DATA_LEN} & \nameref{sec:MEM/WB}.\nameref{MEM/WB:hi_o} & WB 阶段的指令写HI的数据 \\
    in & \labelname{wb_lo_i}\newtag{wb_lo_i}{EX:wb_lo_i} & STD_LOGIC_VECTOR & \nameref{const:REG_DATA_LEN} & \nameref{sec:MEM/WB}.\nameref{MEM/WB:lo_o} & WB 阶段的指令写LO的数据 \\
    in & \labelname{clock_cycle_cnt_i}\newtag{clock_cycle_cnt_i}{EX:clock_cycle_cnt_i} & STD_LOGIC_VECTOR & \nameref{const:ACCU_CNT_LEN} & \nameref{sec:EX/MEM}.\nameref{EX/MEM:clock_cycle_cnt_o} & 进行到了乘累加指令的第几个周期 \\
    in & \labelname{mul_cur_result_i}\newtag{mul_cur_result_i}{EX:mul_cur_result_i} & STD_LOGIC_VECTOR & \nameref{const:DOUBLE_DATA_LEN} & \nameref{sec:EX/MEM}.\nameref{EX/MEM:mul_cur_result_o} & 乘累加指令当前结果 \\
    in & \labelname{is_in_delayslot_i}\newtag{is_in_delayslot_i}{EX:is_in_delayslot_i} & STD_LOGIC & 1 & \nameref{sec:ID/EX}.\nameref{ID/EX:is_in_delayslot_o} & 当前指令是否在延迟槽内 \\
    in & \labelname{link_addr_i}\newtag{link_addr_i}{EX:link_addr_i} & STD_LOGIC_VECTOR & \nameref{const:INST_ADDR_LEN} & \nameref{sec:ID/EX}.\nameref{ID/EX:link_addr_o} & 跳转指令的返回地址 \\
    out & \labelname{reg_wt_en_o}\newtag{reg_wt_en_o}{EX:reg_wt_en_o} & STD_LOGIC & 1 & \nameref{sec:EX/MEM}.\nameref{EX/MEM:reg_wt_en_i} & 寄存器写使能 \\
    out & \labelname{reg_wt_addr_o}\newtag{reg_wt_addr_o}{EX:reg_wt_addr_o} & STD_LOGIC_VECTOR & \nameref{const:REG_ADDR_LEN} & \nameref{sec:EX/MEM}.\nameref{EX/MEM:reg_wt_addr_i} & 寄存器写地址 \\
    out & \labelname{reg_wt_data_o}\newtag{reg_wt_data_o}{EX:reg_wt_data_o} & STD_LOGIC_VECTOR & \nameref{const:REG_DATA_LEN} & \nameref{sec:EX/MEM}.\nameref{EX/MEM:reg_wt_data_i} & 寄存器写数据 \\
    out & \labelname{is_load_store_o}\newtag{is_load_store_o}{EX:is_load_store_o} & STD_LOGIC & 1 & \nameref{sec:EX/MEM}.\nameref{EX/MEM:is_load_store_i} & 当前指令是否为访存指令 \\
    out & \labelname{funct_o}\newtag{funct_o}{EX:funct_o} & STD_LOGIC_VECTOR & \nameref{const:FUNCT_LEN} & \nameref{sec:EX/MEM}.\nameref{EX/MEM:funct_i} & 访存指令子操作类型 \\
    out & \labelname{load_store_addr_o}\newtag{load_store_addr_o}{EX:load_store_addr_o} & STD_LOGIC_VECTOR & \nameref{const:ADDR_LEN} & \nameref{sec:EX/MEM}.\nameref{EX/MEM:load_store_addr_i} & 访存指令访问的地址 \\
    out & \labelname{store_data_o}\newtag{store_data_o}{EX:store_data_o} & STD_LOGIC_VECTOR & \nameref{const:DATA_LEN} & \nameref{sec:EX/MEM}.\nameref{EX/MEM:store_data_i} & store 指令要存储的数据 \\
    out & \labelname{hilo_en_o}\newtag{hilo_en_o}{EX:hilo_en_o} & STD_LOGIC & 1 & \nameref{sec:EX/MEM}.\nameref{EX/MEM:hilo_en_i} & 写HILO 使能 \\
    out & \labelname{hi_o}\newtag{hi_o}{EX:hi_o} & STD_LOGIC_VECTOR & \nameref{const:REG_DATA_LEN} & \nameref{sec:EX/MEM}.\nameref{EX/MEM:hi_i} & 写HI 数据 \\
    out & \labelname{lo_o}\newtag{lo_o}{EX:lo_o} & STD_LOGIC_VECTOR & \nameref{const:REG_DATA_LEN} & \nameref{sec:EX/MEM}.\nameref{EX/MEM:lo_i} & 写LO 数据 \\
    out & \labelname{pause_o}\newtag{pause_o}{EX:pause_o} & STD_LOGIC & 1 & \nameref{sec:PAUSE_CTRL}.\nameref{PAUSE_CTRL:ex_pause_i} & 是否需要暂停 \\
    out & \labelname{clock_cycle_cnt_o}\newtag{clock_cycle_cnt_o}{EX:clock_cycle_cnt_o} & STD_LOGIC_VECTOR & \nameref{const:ACCU_CNT_LEN} & \nameref{sec:EX/MEM}.\nameref{EX/MEM:clock_cycle_cnt_i} & 进行到了乘累加指令的第几个周期 \\
    out & \labelname{mul_cur_result_o}\newtag{mul_cur_result_o}{EX:mul_cur_result_o} & STD_LOGIC_VECTOR & \nameref{const:DOUBLE_DATA_LEN} & \nameref{sec:EX/MEM}.\nameref{EX/MEM:mul_cur_result_i} & 乘累加指令当前结果 \\
    \end{supertabular}
\end{center}
\FloatBarrier

\section{EX/MEM\label{sec:EX/MEM}}

\subsection{简介}

\FloatBarrier
\subsection{接口定义}

\begin{center}
    \tablecaption{PC的接口}

    \tablefirsthead{
        \toprule
        \multicolumn{1}{l}{方向} &
        \multicolumn{1}{l}{名称} &
        \multicolumn{1}{l}{类型} &
        \multicolumn{1}{l}{宽度} &
        \multicolumn{1}{l}{连接到} &
        \multicolumn{1}{l}{详细描述} \\
        \midrule }

    \tablehead{
        \toprule
        \multicolumn{1}{l}{方向} &
        \multicolumn{1}{l}{名称} &
        \multicolumn{1}{l}{类型} &
        \multicolumn{1}{l}{宽度} &
        \multicolumn{1}{l}{连接到} &
        \multicolumn{1}{l}{详细描述} \\
        \midrule }

    \tabletail{
        \bottomrule
        \multicolumn{6}{c}{接下页} \\}

    \tablelasttail{\bottomrule}

    \small
    \begin{supertabular}{lllllp{2.5cm}}

    in & \labelname{rst}\newtag{rst}{EX/MEM:rst} & STD_LOGIC & 1 & \nameref{sec:MIPS_CPU}.\nameref{MIPS_CPU:rst} & 复位信号 \\
    in & \labelname{clk}\newtag{clk}{EX/MEM:clk} & STD_LOGIC & 1 & \nameref{sec:MIPS_CPU}.\nameref{MIPS_CPU:clk} & 时钟信号 \\
    in & \labelname{reg_wt_en_i}\newtag{reg_wt_en_i}{EX/MEM:reg_wt_en_i} & STD_LOGIC & 1 & \nameref{sec:EX}.\nameref{EX:reg_wt_en_o} & 寄存器写使能 \\
    in & \labelname{reg_wt_addr_i}\newtag{reg_wt_addr_i}{EX/MEM:reg_wt_addr_i} & STD_LOGIC_VECTOR & \nameref{const:REG_ADDR_LEN} & \nameref{sec:EX}.\nameref{EX:reg_wt_addr_o} & 寄存器写地址 \\
    in & \labelname{reg_wt_data_i}\newtag{reg_wt_data_i}{EX/MEM:reg_wt_data_i} & STD_LOGIC_VECTOR & \nameref{const:REG_DATA_LEN} & \nameref{sec:EX}.\nameref{EX:reg_wt_data_o} & 寄存器写数据 \\
    in & \labelname{is_load_store_i}\newtag{is_load_store_i}{EX/MEM:is_load_store_i} & STD_LOGIC & 1 & \nameref{sec:EX}.\nameref{EX:is_load_store_o} & 当前指令是否为访存指令 \\
    in & \labelname{funct_i}\newtag{funct_i}{EX/MEM:funct_i} & STD_LOGIC_VECTOR & \nameref{const:FUNCT_LEN} & \nameref{sec:EX}.\nameref{EX:funct_o} & 访存指令子操作类型 \\
    in & \labelname{load_store_addr_i}\newtag{load_store_addr_i}{EX/MEM:load_store_addr_i} & STD_LOGIC_VECTOR & \nameref{const:ADDR_LEN} & \nameref{sec:EX}.\nameref{EX:load_store_addr_o} & 访存指令访问的地址 \\
    in & \labelname{store_data_i}\newtag{store_data_i}{EX/MEM:store_data_i} & STD_LOGIC_VECTOR & \nameref{const:DATA_LEN} & \nameref{sec:EX}.\nameref{EX:store_data_o} & store指令要存储的数据 \\
    in & \labelname{hilo_en_i}\newtag{hilo_en_i}{EX/MEM:hilo_en_i} & STD_LOGIC & 1 & \nameref{sec:EX}.\nameref{EX:hilo_en_o} & 写HILO 使能 \\
    in & \labelname{hi_i}\newtag{hi_i}{EX/MEM:hi_i} & STD_LOGIC_VECTOR & \nameref{const:REG_DATA_LEN} & \nameref{sec:EX}.\nameref{EX:hi_o} & 写HI 数据 \\
    in & \labelname{lo_i}\newtag{lo_i}{EX/MEM:lo_i} & STD_LOGIC_VECTOR & \nameref{const:REG_DATA_LEN} & \nameref{sec:EX}.\nameref{EX:lo_o} & 写LO 数据 \\
    in & \labelname{pause_i}\newtag{pause_i}{EX/MEM:pause_i} & STD_LOGIC_VECTOR & \nameref{const:CTRL_PAUSE_LEN} & \nameref{sec:PAUSE_CTRL}.\nameref{PAUSE_CTRL:pause_o} & 流水线当前阶段是否需要暂停 \\
    in & \labelname{clock_cycle_cnt_i}\newtag{clock_cycle_cnt_i}{EX/MEM:clock_cycle_cnt_i} & STD_LOGIC_VECTOR & \nameref{const:ACCU_CNT_LEN} & \nameref{sec:EX}.\nameref{EX:clock_cycle_cnt_i} & 进行到了乘累加指令的第几个周期 \\
    in & \labelname{mul_cur_result_i}\newtag{mul_cur_result_i}{EX/MEM:mul_cur_result_i} & STD_LOGIC_VECTOR & \nameref{const:DOUBLE_DATA_LEN} & \nameref{sec:EX}.\nameref{EX:mul_cur_result_o} & 乘累加指令当前结果 \\
    out & \labelname{reg_wt_en_o}\newtag{reg_wt_en_o}{EX/MEM:reg_wt_en_o} & STD_LOGIC & 1 & \nameref{sec:MEM}.\nameref{MEM:reg_wt_en_i} & 寄存器写使能 \\
    out & \labelname{reg_wt_addr_o}\newtag{reg_wt_addr_o}{EX/MEM:reg_wt_addr_o} & STD_LOGIC_VECTOR & \nameref{const:REG_ADDR_LEN} & \nameref{sec:MEM}.\nameref{MEM:reg_wt_addr_i} & 寄存器写地址 \\
    out & \labelname{reg_wt_data_o}\newtag{reg_wt_data_o}{EX/MEM:reg_wt_data_o} & STD_LOGIC_VECTOR & \nameref{const:REG_DATA_LEN} & \nameref{sec:MEM}.\nameref{MEM:reg_wt_data_i} & 寄存器写数据 \\
    out & \labelname{is_load_store_o}\newtag{is_load_store_o}{EX/MEM:is_load_store_o} & STD_LOGIC & 1 & \nameref{sec:MEM}.\nameref{MEM:is_load_store_i} & 当前指令是否为访存指令 \\
    out & \labelname{funct_o}\newtag{funct_o}{EX/MEM:funct_o} & STD_LOGIC_VECTOR & \nameref{const:FUNCT_LEN} & \nameref{sec:MEM}.\nameref{MEM:funct_i} & 访存指令子操作类型 \\
    out & \labelname{load_store_addr_o}\newtag{load_store_addr_o}{EX/MEM:load_store_addr_o} & STD_LOGIC_VECTOR & \nameref{const:ADDR_LEN} & \nameref{sec:MEM}.\nameref{MEM:load_store_addr_i} & 访存指令访问的地址 \\
    out & \labelname{store_data_o}\newtag{store_data_o}{EX/MEM:store_data_o} & STD_LOGIC_VECTOR & \nameref{const:DATA_LEN} & \nameref{sec:MEM}.\nameref{MEM:store_data_i} & store指令要存储的数据 \\
    out & \labelname{hilo_en_o}\newtag{hilo_en_o}{EX/MEM:hilo_en_o} & STD_LOGIC & 1 & \nameref{sec:MEM}.\nameref{MEM:hilo_en_i} & 写HILO 使能 \\
    out & \labelname{hi_o}\newtag{hi_o}{EX/MEM:hi_o} & STD_LOGIC_VECTOR & \nameref{const:REG_DATA_LEN} & \nameref{sec:MEM}.\nameref{MEM:hi_i} & 写HI数据 \\
    out & \labelname{lo_o}\newtag{lo_o}{EX/MEM:lo_o} & STD_LOGIC_VECTOR & \nameref{const:REG_DATA_LEN} & \nameref{sec:MEM}.\nameref{MEM:lo_i} & 写LO数据 \\
    out & \labelname{clock_cycle_cnt_o}\newtag{clock_cycle_cnt_o}{EX/MEM:clock_cycle_cnt_o} & STD_LOGIC_VECTOR & \nameref{const:ACCU_CNT_LEN} & \nameref{sec:EX}.\nameref{EX:clock_cycle_cnt_i} & 进行到了乘累加指令的第几个周期 \\
    out & \labelname{mul_cur_result_o}\newtag{mul_cur_result_o}{EX/MEM:mul_cur_result_o} & STD_LOGIC_VECTOR & \nameref{const:DOUBLE_DATA_LEN} & \nameref{sec:EX}.\nameref{EX:mul_cur_result_o} & 乘累加指令当前结果 \\
    \end{supertabular}
\end{center}
\FloatBarrier

\section{MEM\label{sec:MEM}}

\subsection{简介}

现在正在研究要不要写暂停。但是可能就不写了,因为可以抄别的组的文档。

\FloatBarrier
\subsection{接口定义}
\begin{center}
    \tablecaption{PC的接口}

    \tablefirsthead{
        \toprule
        \multicolumn{1}{l}{方向} &
        \multicolumn{1}{l}{名称} &
        \multicolumn{1}{l}{类型} &
        \multicolumn{1}{l}{宽度} &
        \multicolumn{1}{l}{连接到} &
        \multicolumn{1}{l}{详细描述} \\
        \midrule }

    \tablehead{
        \toprule
        \multicolumn{1}{l}{方向} &
        \multicolumn{1}{l}{名称} &
        \multicolumn{1}{l}{类型} &
        \multicolumn{1}{l}{宽度} &
        \multicolumn{1}{l}{连接到} &
        \multicolumn{1}{l}{详细描述} \\
        \midrule }

    \tabletail{
        \bottomrule
        \multicolumn{6}{c}{接下页} \\}

    \tablelasttail{\bottomrule}

    \small
    \begin{supertabular}{llllp{3.5cm}p{3cm}}

    in & \labelname{rst}\newtag{rst}{MEM:rst} & STD_LOGIC & 1 & \nameref{sec:MIPS_CPU}.\nameref{MIPS_CPU:rst} & 复位信号 \\
    in & \labelname{reg_wt_en_i}\newtag{reg_wt_en_i}{MEM:reg_wt_en_i} & STD_LOGIC & 1 & \nameref{sec:EX/MEM}.\nameref{EX/MEM:reg_wt_en_o} & 寄存器写使能 \\
    in & \labelname{reg_wt_addr_i}\newtag{reg_wt_addr_i}{MEM:reg_wt_addr_i} & STD_LOGIC_VECTOR & \nameref{const:REG_ADDR_LEN} & \nameref{sec:EX/MEM}.\nameref{EX/MEM:reg_wt_addr_o} & 寄存器写地址 \\
    in & \labelname{reg_wt_data_i}\newtag{reg_wt_data_i}{MEM:reg_wt_data_i} & STD_LOGIC_VECTOR & \nameref{const:REG_DATA_LEN} & \nameref{sec:EX/MEM}.\nameref{EX/MEM:reg_wt_data_o} & 寄存器写数据 \\
    in & \labelname{ram_rd_data_i}\newtag{ram_rd_data_i}{MEM:ram_rd_data_i} & STD_LOGIC_VECTOR & \nameref{const:DATA_LEN} & \nameref{sec:MIPS_CPU}.\nameref{MIPS_CPU:ram_data_o} & 从外部读取的数据 \\
    in & \labelname{is_load_store_i}\newtag{is_load_store_i}{MEM:is_load_store_i} & STD_LOGIC & 1 & \nameref{sec:EX/MEM}.\nameref{EX/MEM:is_load_store_o} & 当前指令是否为访存指令 \\
    in & \labelname{funct_i}\newtag{funct_i}{MEM:funct_i} & STD_LOGIC_VECTOR & \nameref{const:FUNCT_LEN} & \nameref{sec:EX/MEM}.\nameref{EX/MEM:funct_o} & 访存指令子操作类型 \\
    in & \labelname{load_store_addr_i}\newtag{load_store_addr_i}{MEM:load_store_addr_i} & STD_LOGIC_VECTOR & \nameref{const:ADDR_LEN} & \nameref{sec:EX/MEM}.\nameref{EX/MEM:load_store_addr_o} & 访存指令访问的地址 \\
    in & \labelname{store_data_i}\newtag{store_data_i}{MEM:store_data_i} & STD_LOGIC_VECTOR & \nameref{const:DATA_LEN} & \nameref{sec:EX/MEM}.\nameref{EX/MEM:store_data_o} & store指令要存储的数据 \\
    in & \labelname{hilo_en_i}\newtag{hilo_en_i}{MEM:hilo_en_i} & STD_LOGIC & 1 & \nameref{sec:EX/MEM}.\nameref{EX/MEM:hilo_en_o} & 写HILO 使能 \\
    in & \labelname{hi_i}\newtag{hi_i}{MEM:hi_i} & STD_LOGIC_VECTOR & \nameref{const:REG_DATA_LEN} & \nameref{sec:EX/MEM}.\nameref{EX/MEM:hi_o} & 写HI数据 \\
    in & \labelname{lo_i}\newtag{lo_i}{MEM:lo_i} & STD_LOGIC_VECTOR & \nameref{const:REG_DATA_LEN} & \nameref{sec:EX/MEM}.\nameref{EX/MEM:lo_o} & 写LO数据 \\
    out & \labelname{reg_wt_en_o}\newtag{reg_wt_en_o}{MEM:reg_wt_en_o} & STD_LOGIC & 1 & \nameref{sec:MEM/WB}.\nameref{MEM/WB:reg_wt_en_i} & 寄存器写使能 \\
    out & \labelname{reg_wt_addr_o}\newtag{reg_wt_addr_o}{MEM:reg_wt_addr_o} & STD_LOGIC_VECTOR & \nameref{const:REG_ADDR_LEN} & \nameref{sec:MEM/WB}.\nameref{MEM/WB:reg_wt_addr_i} & 寄存器写地址 \\
    out & \labelname{reg_wt_data_o}\newtag{reg_wt_data_o}{MEM:reg_wt_data_o} & STD_LOGIC_VECTOR & \nameref{const:REG_DATA_LEN} & \nameref{sec:MEM/WB}.\nameref{MEM/WB:reg_wt_data_i} & 寄存器写数据 \\
    out & \labelname{ram_en_o}\newtag{ram_en_o}{MEM:ram_en_o} & STD_LOGIC & 1 &\nameref{sec:MIPS_CPU}.\nameref{MIPS_CPU:ram_en_o} & RAM 读写使能 \\
    out & \labelname{ram_is_read_o}\newtag{ram_is_read_o}{MEM:ram_is_read_o} & STD_LOGIC & 1 & \nameref{sec:MIPS_CPU}.\nameref{MIPS_CPU:ram_is_read_o} & RAM 是否为读 \\
    out & \labelname{ram_addr_o}\newtag{ram_addr_o}{MEM:ram_addr_o} & STD_LOGIC_VECTOR & \nameref{const:ADDR_LEN} & \nameref{sec:MIPS_CPU}.\nameref{MIPS_CPU:ram_addr_o} & RAM的访问地址 \\
    out & \labelname{ram_data_o}\newtag{ram_data_o}{MEM:ram_data_o} & STD_LOGIC_VECTOR & \nameref{const:DATA_LEN} & \nameref{sec:MIPS_CPU}.\nameref{MIPS_CPU:ram_data_o} & RAM的写数据 \\
    out & \labelname{ram_data_sel_o}\newtag{ram_data_sel_o}{MEM:ram_data_sel_o} & STD_LOGIC_VECTOR & \nameref{const:BYTE_IN_DATA} & \nameref{sec:MIPS_CPU}.\nameref{MIPS_CPU:ram_data_sel_o} & RAM数据选择 \\
    out & \labelname{hilo_en_o}\newtag{hilo_en_o}{MEM:hilo_en_o} & STD_LOGIC & 1 & \nameref{sec:MEM/WB}.\nameref{MEM/WB:hilo_en_i}, \nameref{sec:EX}.\nameref{EX:mem_hilo_en_i} & 写HILO 使能 \\
    out & \labelname{hi_o}\newtag{hi_o}{MEM:hi_o} & STD_LOGIC_VECTOR & \nameref{const:REG_DATA_LEN} & \nameref{sec:MEM/WB}.\nameref{MEM/WB:hi_i}, \nameref{sec:EX}.\nameref{EX:mem_hi_i} & 写HI数据 \\
    out & \labelname{lo_o}\newtag{lo_o}{MEM:lo_o} & STD_LOGIC_VECTOR & \nameref{const:REG_DATA_LEN} & \nameref{sec:MEM/WB}.\nameref{MEM/WB:lo_i}, \nameref{sec:EX}.\nameref{EX:mem_lo_i} & 写LO数据 \\
    \end{supertabular}
\end{center}
\FloatBarrier

\section{MEM/WB\label{sec:MEM/WB}}

\subsection{简介}

\FloatBarrier
\subsection{接口定义}

\begin{center}
    \tablecaption{PC的接口}

    \tablefirsthead{
        \toprule
        \multicolumn{1}{l}{方向} &
        \multicolumn{1}{l}{名称} &
        \multicolumn{1}{l}{类型} &
        \multicolumn{1}{l}{宽度} &
        \multicolumn{1}{l}{连接到} &
        \multicolumn{1}{l}{详细描述} \\
        \midrule }

    \tablehead{
        \toprule
        \multicolumn{1}{l}{方向} &
        \multicolumn{1}{l}{名称} &
        \multicolumn{1}{l}{类型} &
        \multicolumn{1}{l}{宽度} &
        \multicolumn{1}{l}{连接到} &
        \multicolumn{1}{l}{详细描述} \\
        \midrule }

    \tabletail{
        \bottomrule
        \multicolumn{6}{c}{接下页} \\}

    \tablelasttail{\bottomrule}

    \small
    \begin{supertabular}{lllllp{2cm}}

    in & \labelname{rst}\newtag{rst}{MEM/WB:rst} & STD_LOGIC & 1 & \nameref{sec:MIPS_CPU}.\nameref{MIPS_CPU:rst} & 复位信号 \\
    in & \labelname{clk}\newtag{clk}{MEM/WB:clk} & STD_LOGIC & 1 & \nameref{sec:MIPS_CPU}.\nameref{MIPS_CPU:clk} & 时钟信号 \\
    in & \labelname{reg_wt_en_i}\newtag{reg_wt_en_i}{MEM/WB:reg_wt_en_i} & STD_LOGIC & 1 & \nameref{sec:MEM}.\nameref{MEM:reg_wt_en_o} & 寄存器写使能 \\
    in & \labelname{reg_wt_addr_i}\newtag{reg_wt_addr_i}{MEM/WB:reg_wt_addr_i} & STD_LOGIC_VECTOR & \nameref{const:REG_ADDR_LEN} & \nameref{sec:MEM}.\nameref{MEM:reg_wt_addr_o} & 寄存器写地址 \\
    in & \labelname{reg_wt_data_i}\newtag{reg_wt_data_i}{MEM/WB:reg_wt_data_i} & STD_LOGIC_VECTOR & \nameref{const:REG_DATA_LEN} & \nameref{sec:MEM}.\nameref{MEM:reg_wt_data_o} & 寄存器写数据 \\
    in & \labelname{hilo_en_i}\newtag{hilo_en_i}{MEM/WB:hilo_en_i} & STD_LOGIC & 1 & \nameref{sec:MEM}.\nameref{MEM:hilo_en_o} & 写HILO 使能 \\
    in & \labelname{hi_i}\newtag{hi_i}{MEM/WB:hi_i} & STD_LOGIC_VECTOR & \nameref{const:REG_DATA_LEN} & \nameref{sec:MEM}.\nameref{MEM:hi_o} & 写HI数据 \\
    in & \labelname{lo_i}\newtag{lo_i}{MEM/WB:lo_i} & STD_LOGIC_VECTOR & \nameref{const:REG_DATA_LEN} & \nameref{sec:MEM}.\nameref{MEM:lo_o} & 写LO数据 \\
    in & \labelname{pause_i}\newtag{pause_i}{MEM/WB:pause_i} & STD_LOGIC_VECTOR & \nameref{const:CTRL_PAUSE_LEN} & \nameref{sec:PAUSE_CTRL}.\nameref{PAUSE_CTRL:pause_o} & 是否暂停 \\
    out & \labelname{reg_wt_en_o}\newtag{reg_wt_en_o}{MEM/WB:reg_wt_en_o} & STD_LOGIC & 1 & \nameref{sec:REGISTERS}.\nameref{REGISTERS:reg_wt_en_i} & 寄存器写使能 \\
    out & \labelname{reg_wt_addr_o}\newtag{reg_wt_addr_o}{MEM/WB:reg_wt_addr_o} & STD_LOGIC_VECTOR & \nameref{const:REG_ADDR_LEN} & \nameref{sec:REGISTERS}.\nameref{REGISTERS:reg_wt_addr_i} & 寄存器写地址 \\
    out & \labelname{reg_wt_data_o}\newtag{reg_wt_data_o}{MEM/WB:reg_wt_data_o} & STD_LOGIC_VECTOR & \nameref{const:REG_DATA_LEN} & \nameref{sec:REGISTERS}.\nameref{REGISTERS:reg_wt_data_i} & 寄存器写数据 \\
    out & \labelname{hilo_en_o}\newtag{hilo_en_o}{MEM/WB:hilo_en_o} & STD_LOGIC & 1 & \nameref{sec:HI_LO}.\nameref{HI_LO:en}, \nameref{sec:EX}.\nameref{EX:wb_hilo_en_i} & 写HILO 使能 \\
    out & \labelname{hi_o}\newtag{hi_o}{MEM/WB:hi_o} & STD_LOGIC_VECTOR & \nameref{const:REG_DATA_LEN} & \nameref{sec:HI_LO}.\nameref{HI_LO:hi_i}, \nameref{sec:EX}.\nameref{EX:wb_hi_i} & 写HI数据 \\
    out & \labelname{lo_o}\newtag{lo_o}{MEM/WB:lo_o} & STD_LOGIC_VECTOR & \nameref{const:REG_DATA_LEN} & \nameref{sec:HI_LO}.\nameref{HI_LO:lo_i}, \nameref{sec:EX}.\nameref{EX:wb_lo_i} & 写LO 数据 \\
    \end{supertabular}
\end{center}
\FloatBarrier

\section{REGISTERS\label{sec:REGISTERS}}

\subsection{简介}

\FloatBarrier
\subsection{接口定义}

\begin{center}
    \tablecaption{REGISTERS模块的接口}

    \tablefirsthead{
        \toprule
        \multicolumn{1}{l}{方向} &
        \multicolumn{1}{l}{名称} &
        \multicolumn{1}{l}{类型} &
        \multicolumn{1}{l}{宽度} &
        \multicolumn{1}{l}{连接到} &
        \multicolumn{1}{l}{详细描述} \\
        \midrule }

    \tablehead{
        \toprule
        \multicolumn{1}{l}{方向} &
        \multicolumn{1}{l}{名称} &
        \multicolumn{1}{l}{类型} &
        \multicolumn{1}{l}{宽度} &
        \multicolumn{1}{l}{连接到} &
        \multicolumn{1}{l}{详细描述} \\
        \midrule }

    \tabletail{
        \bottomrule
        \multicolumn{6}{c}{接下页} \\}

    \tablelasttail{\bottomrule}

    \small
    \begin{supertabular}{llllp{4cm}p{3cm}}

    in & \labelname{rst}\newtag{rst}{REGISTERS:rst} & STD_LOGIC & 1 & \nameref{sec:MIPS_CPU}.\nameref{MIPS_CPU:rst} & 复位信号    \\
    in & \labelname{clk}\newtag{clk}{REGISTERS:clk} & STD_LOGIC & 1 & \nameref{sec:MIPS_CPU}.\nameref{MIPS_CPU:clk} & 时钟信号    \\
    in & \labelname{reg_rd_en_1_i}\newtag{reg_rd_en_1_i}{REGISTERS:reg_rd_en_1_i} & STD_LOGIC & 1 & \nameref{sec:ID}.\nameref{ID:reg_rd_en_1_o} & 寄存器1 读使能 \\
    in & \labelname{reg_rd_en_2_i}\newtag{reg_rd_en_2_i}{REGISTERS:reg_rd_en_2_i} & STD_LOGIC & 1 & \nameref{sec:ID}.\nameref{ID:reg_rd_en_2_o} & 寄存器2读使能 \\
    in & \labelname{reg_rd_addr_1_i}\newtag{reg_rd_addr_1_i}{REGISTERS:reg_rd_addr_1_i} & STD_LOGIC_VECTOR & \nameref{const:REG_ADDR_LEN} & \nameref{sec:ID}.\nameref{ID:reg_rd_addr_1_o} & 寄存器1读地址 \\
    in & \labelname{reg_rd_addr_2_i}\newtag{reg_rd_addr_2_i}{REGISTERS:reg_rd_addr_2_i} & STD_LOGIC_VECTOR & \nameref{const:REG_ADDR_LEN} & \nameref{sec:ID}.\nameref{ID:reg_rd_addr_2_o} & 寄存器2读地址 \\
    in & \labelname{reg_wt_en_i}\newtag{reg_wt_en_i}{REGISTERS:reg_wt_en_i} & STD_LOGIC & 1 & \nameref{sec:MEM/WB}.\nameref{MEM/WB:reg_wt_en_o} & 寄存器写使能 \\
    in & \labelname{reg_wt_addr_i}\newtag{reg_wt_addr_i}{REGISTERS:reg_wt_addr_i} & STD_LOGIC_VECTOR & \nameref{const:REG_ADDR_LEN} & \nameref{sec:MEM/WB}.\nameref{MEM/WB:reg_wt_addr_o} & 寄存器写地址 \\
    in & \labelname{reg_wt_data_i}\newtag{reg_wt_data_i}{REGISTERS:reg_wt_data_i} & STD_LOGIC_VECTOR & \nameref{const:REG_DATA_LEN} & \nameref{sec:MEM/WB}.\nameref{MEM/WB:reg_wt_data_o} & 寄存器写数据 \\
    out & \labelname{reg_rd_data_1_o}\newtag{reg_rd_data_1_o}{REGISTERS:reg_rd_data_1_o} & STD_LOGIC_VECTOR & \nameref{const:REG_DATA_LEN} & \nameref{sec:ID}.\nameref{ID:reg_rd_data_1_i} & 寄存器1读出数据 \\
    out & \labelname{reg_rd_data_2_o}\newtag{reg_rd_data_2_o}{REGISTERS:reg_rd_data_2_o} & STD_LOGIC_VECTOR & \nameref{const:REG_DATA_LEN} & \nameref{sec:ID}.\nameref{ID:reg_rd_data_2_i} & 寄存器2读出数据 \\

    \end{supertabular}
\end{center}
\FloatBarrier

\section{HI_LO}
\label{sec:HI_LO}

\subsection{简介}

\FloatBarrier
\subsection{接口定义}

\begin{center}
    \tablecaption{HILO模块的接口}

    \tablefirsthead{
        \toprule
        \multicolumn{1}{l}{方向} &
        \multicolumn{1}{l}{名称} &
        \multicolumn{1}{l}{类型} &
        \multicolumn{1}{l}{宽度} &
        \multicolumn{1}{l}{连接到} &
        \multicolumn{1}{l}{详细描述} \\
        \midrule }

    \tablehead{
        \toprule
        \multicolumn{1}{l}{方向} &
        \multicolumn{1}{l}{名称} &
        \multicolumn{1}{l}{类型} &
        \multicolumn{1}{l}{宽度} &
        \multicolumn{1}{l}{连接到} &
        \multicolumn{1}{l}{详细描述} \\
        \midrule }

    \tabletail{
        \bottomrule
        \multicolumn{6}{c}{接下页} \\}

    \tablelasttail{\bottomrule}

    \small
    \begin{supertabular}{lllllp{2cm}}

    in & \labelname{rst}\newtag{rst}{HI_LO:rst} & STD_LOGIC & 1 & \nameref{sec:MIPS_CPU}.\nameref{MIPS_CPU:rst} & 复位信号 \\
    in & \labelname{clk}\newtag{clk}{HI_LO:clk} & STD_LOGIC & 1 & \nameref{sec:MIPS_CPU}.\nameref{MIPS_CPU:clk} & 时钟信号 \\
    in & \labelname{en}\newtag{en}{HI_LO:en} & STD_LOGIC & 1 & \nameref{sec:MEM/WB}.\nameref{MEM/WB:hilo_en_o} & 使能 \\
    in & \labelname{hi_i}\newtag{hi_i}{HI_LO:hi_i} & STD_LOGIC_VECTOR & \nameref{const:REG_DATA_LEN} & \nameref{sec:MEM/WB}.\nameref{MEM/WB:hi_o} & HI \\
    in & \labelname{lo_i}\newtag{lo_i}{HI_LO:lo_i} & STD_LOGIC_VECTOR & \nameref{const:REG_DATA_LEN} & \nameref{sec:MEM/WB}.\nameref{MEM/WB:lo_o} & LO \\
    out & \labelname{hi_o}\newtag{hi_o}{HI_LO:hi_o} & STD_LOGIC_VECTOR & \nameref{const:REG_DATA_LEN} & \nameref{sec:EX}.\nameref{EX:hi_i} & HI \\
    out & \labelname{lo_o}\newtag{lo_o}{HI_LO:lo_o} & STD_LOGIC_VECTOR & \nameref{const:REG_DATA_LEN} & \nameref{sec:EX}.\nameref{EX:hi_o} & LO \\

    \end{supertabular}
\end{center}
\FloatBarrier

\section{PAUSE_CTRL}
\label{sec:PAUSE_CTRL}

\subsection{简介}

\FloatBarrier
\subsection{接口定义}

\begin{center}
    \tablecaption{PAUSE_CTRL模块的接口}

    \tablefirsthead{
        \toprule
        \multicolumn{1}{l}{方向} &
        \multicolumn{1}{l}{名称} &
        \multicolumn{1}{l}{类型} &
        \multicolumn{1}{l}{宽度} &
        \multicolumn{1}{l}{连接到} &
        \multicolumn{1}{l}{详细描述} \\
        \midrule }

    \tablehead{
        \toprule
        \multicolumn{1}{l}{方向} &
        \multicolumn{1}{l}{名称} &
        \multicolumn{1}{l}{类型} &
        \multicolumn{1}{l}{宽度} &
        \multicolumn{1}{l}{连接到} &
        \multicolumn{1}{l}{详细描述} \\
        \midrule }

    \tabletail{
        \bottomrule
        \multicolumn{6}{c}{接下页} \\}

    \tablelasttail{\bottomrule}

    \small
    \begin{supertabular}{llllp{5cm}p{2cm}}

    in & \labelname{rst}\newtag{rst}{PAUSE_CTRL:rst} & STD_LOGIC & 1 & \nameref{sec:MIPS_CPU}.\nameref{EX:extended_offset_i} & 复位信号 \\
    in & \labelname{id_pause_i}\newtag{id_pause_i}{PAUSE_CTRL:id_pause_i} & STD_LOGIC & 1 & \nameref{sec:ID}.\nameref{ID:pause_o} & ID 模块是否暂停 \\
    in & \labelname{ex_pause_i}\newtag{ex_pause_i}{PAUSE_CTRL:ex_pause_i} & STD_LOGIC & 1 & \nameref{sec:EX}.\nameref{EX:pause_o} & EX 模块是否暂停 \\
    out & \labelname{pause_o}\newtag{pause_o}{PAUSE_CTRL:pause_o} & STD_LOGIC_VECTOR & \nameref{const:CTRL_PAUSE_LEN} & \nameref{sec:PC}.\nameref{PC:pause_i}, \nameref{sec:IF/ID}.\nameref{IF/ID:pause_i}, \nameref{sec:ID/EX}.\nameref{ID/EX:pause_i}, \nameref{sec:EX/MEM}.\nameref{EX/MEM:pause_i}, \nameref{sec:MEM/WB}.\nameref{MEM/WB:pause_i} & 各模块是否暂停 \\

    \end{supertabular}
\end{center}
\FloatBarrier

\section{MIPS_CPU}
\label{sec:MIPS_CPU}

\subsection{简介}

\FloatBarrier
\subsection{接口定义}

\begin{center}
    \tablecaption{MIPS_CPU模块的接口}

    \tablefirsthead{
        \toprule
        \multicolumn{1}{l}{方向} &
        \multicolumn{1}{l}{名称} &
        \multicolumn{1}{l}{类型} &
        \multicolumn{1}{l}{宽度} &
        \multicolumn{1}{l}{连接到} &
        \multicolumn{1}{l}{详细描述} \\
        \midrule }

    \tablehead{
        \toprule
        \multicolumn{1}{l}{方向} &
        \multicolumn{1}{l}{名称} &
        \multicolumn{1}{l}{类型} &
        \multicolumn{1}{l}{宽度} &
        \multicolumn{1}{l}{连接到} &
        \multicolumn{1}{l}{详细描述} \\
        \midrule }

    \tabletail{
        \bottomrule
        \multicolumn{6}{c}{接下页} \\}

    \tablelasttail{\bottomrule}

    \small
    \begin{supertabular}{llllp{4.5cm}p{2.5cm}}

    in & \labelname{rst}\newtag{rst}{MIPS_CPU:rst} & STD_LOGIC & 1 & \nameref{sec:PC}.\nameref{PC:rst}, \nameref{sec:IF/ID}.\nameref{IF/ID:rst}, \nameref{sec:ID}.\nameref{ID:rst}, \nameref{sec:ID/EX}.\nameref{ID/EX:rst}, \nameref{sec:EX}.\nameref{EX:rst}, \nameref{sec:EX/MEM}.\nameref{EX/MEM:rst}, \nameref{sec:MEM}.\nameref{MEM:rst}, \nameref{sec:MEM/WB}.\nameref{MEM/WB:rst}, \nameref{sec:REGISTERS}.\nameref{REGISTERS:rst}, \nameref{sec:HI_LO}.\nameref{HI_LO:rst}, \nameref{sec:PAUSE_CTRL}.\nameref{PAUSE_CTRL:rst} & 复位信号 \\
    in & \labelname{clk}\newtag{clk}{MIPS_CPU:clk} & STD_LOGIC & 1 & \nameref{sec:PC}.\nameref{PC:clk}, \nameref{sec:IF/ID}.\nameref{IF/ID:clk}, \nameref{sec:ID/EX}.\nameref{ID/EX:clk}, \nameref{sec:EX/MEM}.\nameref{EX/MEM:clk}, \nameref{sec:MEM/WB}.\nameref{MEM/WB:clk}, \nameref{sec:REGISTERS}.\nameref{REGISTERS:clk}, \nameref{sec:HI_LO}.\nameref{HI_LO:clk} & 时钟信号 \\
    in & \labelname{inst_i}\newtag{inst_i}{MIPS_CPU:inst_i} & STD_LOGIC_VECTOR & \nameref{const:INST_LEN} & \nameref{sec:IF/ID}.\nameref{IF/ID:inst_i}, ROM & 输入指令 \\
    in & \labelname{ram_rd_data_i}\newtag{ram_rd_data_i}{MIPS_CPU:ram_rd_data_i} & STD_LOGIC_VECTOR & \nameref{const:INST_LEN} & \nameref{sec:MEM}.\nameref{MEM:ram_rd_data_i}, RAM & 输入数据 \\
    out & \labelname{rom_en_o}\newtag{rom_en_o}{MIPS_CPU:rom_en_o} & STD_LOGIC & 1 & \nameref{sec:PC}.\nameref{PC:en_o}, ROM & ROM 使能 \\
    out & \labelname{rom_addr_o}\newtag{rom_addr_o}{MIPS_CPU:rom_addr_o} & STD_LOGIC_VECTOR & \nameref{const:INST_LEN} & \nameref{sec:PC}.\nameref{PC:pc_o}, ROM & 指令地址 \\
    out & \labelname{ram_en_o}\newtag{ram_en_o}{MIPS_CPU:ram_en_o} & STD_LOGIC & 1 & \nameref{sec:MEM}.\nameref{MEM:ram_en_o}, RAM & RAM 使能 \\
    out & \labelname{ram_is_read_o}\newtag{ram_is_read_o}{MIPS_CPU:ram_is_read_o} & STD_LOGIC & 1 & \nameref{sec:MEM}.\nameref{MEM:ram_is_read_o}, RAM & RAM是否是读 \\
    out & \labelname{ram_addr_o}\newtag{ram_addr_o}{MIPS_CPU:ram_addr_o} & STD_LOGIC_VECTOR & \nameref{const:INST_LEN} & \nameref{sec:MEM}.\nameref{MEM:ram_addr_o}, RAM & 读写RAM 的地址 \\
    out & \labelname{ram_data_o}\newtag{ram_data_o}{MIPS_CPU:ram_data_o} & STD_LOGIC_VECTOR & \nameref{const:DATA_LEN} & \nameref{sec:MEM}.\nameref{MEM:ram_data_o}, RAM & 写RAM 的数据 \\
    out & \labelname{ram_data_sel_o}\newtag{ram_data_sel_o}{MIPS_CPU:ram_data_sel_o} & STD_LOGIC_VECTOR & \nameref{const:BYTE_IN_DATA} & \nameref{sec:MEM}.\nameref{MEM:ram_data_sel_o}, RAM & 读写内容选择 \\

    \end{supertabular}
\end{center}
\FloatBarrier

\appendix
\section{常数和宏定义}

\subsection{INTEGER类型的常数}
\begin{center}
    \tablecaption{INTEGER类型的常数}

    \tablefirsthead{
        \toprule
        \multicolumn{1}{l}{名称} &
        \multicolumn{1}{l}{内容} &
        \multicolumn{1}{l}{详细描述} \\
        \midrule }

    \tablehead{
        \toprule
        \multicolumn{1}{l}{名称} &
        \multicolumn{1}{l}{内容} &
        \multicolumn{1}{l}{详细描述} \\
        \midrule }

    \tabletail{
        \bottomrule
        \multicolumn{3}{c}{接下页} \\}

    \tablelasttail{\bottomrule}

    \small
    \begin{supertabular}{lll}

    \labelname{INST_ADDR_LEN}\newtag{INST_ADDR_LEN}{const:INST_ADDR_LEN} & 32 & 指令的地址长度 \\
    \labelname{ADDR_LEN}\newtag{ADDR_LEN}{const:ADDR_LEN} & 32 & 普通的地址长度 \\
    \labelname{INST_LEN}\newtag{INST_LEN}{const:INST_LEN} & 32 & 指令长度 \\
    \labelname{REG_ADDR_LEN}\newtag{REG_ADDR_LEN}{const:REG_ADDR_LEN} & 5 & 寄存器地址长度 \\
    \labelname{REG_DATA_LEN}\newtag{REG_DATA_LEN}{const:REG_DATA_LEN} & 32 & 寄存器内数据长度 \\
    \labelname{DATA_LEN}\newtag{DATA_LEN}{const:DATA_LEN} & 32 & 一般的数据长度 \\
    \labelname{DOUBLE_DATA_LEN}\newtag{DOUBLE_DATA_LEN}{const:DOUBLE_DATA_LEN} & 64 & 乘法结果的数据长度 \\
    \labelname{CTRL_PAUSE_LEN}\newtag{CTRL_PAUSE_LEN}{const:CTRL_PAUSE_LEN} & 6 & 暂停控制数据长度 \\
    \labelname{OP_LEN}\newtag{OP_LEN}{const:OP_LEN} & 6 & 指令操作码长度 \\
    \labelname{FUNCT_LEN}\newtag{FUNCT_LEN}{const:FUNCT_LEN} & 6 & 指令子操作码长度 \\
    \labelname{ACCU_CNT_LEN}\newtag{ACCU_CNT_LEN}{const:ACCU_CNT_LEN} & 2 & 乘累加/减指令周期数 \\
    \labelname{BYTE_IN_DATA}\newtag{BYTE_IN_DATA}{const:BYTE_IN_DATA} & 4 & 一个普通的数据中有多少字节 \\

    \end{supertabular}
\end{center}
\end{document}
